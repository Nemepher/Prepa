\documentclass[a4paper]{article}
\usepackage[top=2cm, bottom=2cm, left=2cm, right=2cm]{geometry}
\usepackage{enumitem}
\usepackage{fancyhdr}
\usepackage{lmodern}
\usepackage{indentfirst}
\setenumerate[1]{label=(\Roman*)}
\setenumerate[2]{label=(\alph*)}

\newcommand{\cav}{\textit{Les Cavaliers }}
\newcommand{\adf}{\textit{L'Assemblée des femmes }}
\newcommand{\dda}{\textit{De la démocratie en Amérique }}
\newcommand{\cca}{\textit{Le Complot contre l'Amérique }}

\begin{document}
\pagestyle{fancy}
\fancyhead[L]{Augustin Albert}
\fancyhead[R]{MPSI2 --- 2019/2020}
\fancyhead[C]{Dissertation Rousseau}
	
	Thomas More décrit dans la deuxième partie de son livre \textit{L'Utopie} sa vision d'une société idéale. Celle-ci prend la forme d'une île : Utopie. Les institutions y sont contraires a ce que l'on pourrait imaginer, nettement autoritaires : surveillance généralisée, nombreuses interdictions, contrôle de la population... La démocratie serait elle peu adaptée à cette société idéale ? Il semble pour More qu'une forme de contrôle soit nécessaire. Les hommes, imparfaits, devraient être encadrés pour éviter la déviance. Rousseau, en 1762, semble partager cet avis à travers \textit{Du contrat social} : << S'il y avait un peuple de dieux, il se gouvernerait démocratiquement. Un gouvernement si parfait ne convient pas à des hommes >>. Rousseau imagine un peuple fictif composé d'êtres divins: ces êtres seraient alors omniscients, et en toute conscience font le choix de la démocratie pour se gouverner. Mais si Rousseau semble présenter la démocratie comme le parfait gouvernement, celui-ci nécessite un peuple tout aussi parfait, et notamment infiniment bienveillant. Rousseau oppose ces dieux aux hommes, ne jouissant pas des mêmes facultés : une démocratie ne leur convient pas. Quelles sont les qualités dont manqueraient alors les hommes : une égalité absolue entre tous, la vertu ? Et quelles caractéristiques de la démocratie seraient alors inadaptées à l'homme ? Il pourrait s'agir de limitations techniques : dès qu'un peuple est trop nombreux, il est impossible que tous dirigent de concert. Ces limitations pourraient aussi concerner l'engagement et l'implication attendue de l'homme démocratique. Tous n'en seraient pas capables. Pire, l'homme est peut être fondamentalement mauvais, et a besoin d'être contrôlé : La démocratie offre trop de libertés. Par extension, la démocratie non plus n'est peut être pas parfaite: si elle convient à des dieux uniquement, c'est peut être qu'elle est pleine de failles ou de défauts et que seul des dieux seraient capables de résister à la tentation d'en abuser au profit de leur intérêt personnel. Peut on alors concilier l'imperfection humaine et la démocratie ? Nous répondrons à cette question à la lumière de trois auteurs de l'Antiquité jusqu'à aujourd'hui, Aristophane, Tocqueville et Roth et de leur œuvres respective : \cav (-420), \adf (-392), \dda (Tome II, \(4^{e}\) partie, 1840) et \cca (2004). Nous verrons dans un premier temps que les défaut humains vont à contre-sens de la démocratie. Puis nous verrons que pour autant, des lois sacrifiant la liberté pour mieux contrôler le peuple et améliorer la condition de tous ne sont pas des solutions. Enfin nous verrons que les institutions démocratiques permettent de contrer l'imperfection humaine.\\


	La démocratie n'est effectivement pas adaptée aux hommes. Ceux-ci sont imparfaits. Ils cumulent les vices et doivent être encadrés afin de garantir la sécurité et la stabilité d'un peuple. 
	Tout d'abord, le fatalisme et l'inaction du peuple empêchent le bon fonctionnement de la démocratie. Ce type de gouvernement nécessite une implication positive du peuple. Cette paresse est mis en scène dans \cav : les deux serviteurs de Démos, lorsque confrontés au problème du Paphlagonien, commencent par envisager le suicide au sang de Taureau, envisagent de s'en remettre aux dieux ou de <<décamper>>. Les efforts se font à contrecœur, et lorsque une solution se présente en la personne du Charcutier, ils lui donnent des instructions et le laissent travailler seul. De même, dans \adf, Blépiros est si peu actif dans le vie démocratique qu'il manque l'heure d'ouverture de l'Assemblée après avoir trop dormi. Toqueville note aussi ce comportement de l'Homme démocratique dans \dda : Celui-ci est gagné par une paresse intellectuelle et morale, une sorte de << laisser faire >>. Il ajoute que <<l'intelligence des peuples démocratiques reçoit avec délices les idées simples et générales>>. Ainsi, dans \cca, les élections présidentielle, opposant l'aviateur Lindbergh défendant une position neutre des États-Unis vis à vis de l'Allemagne nazi et l'ancien présidant Roosvelt affirmant la nécessité d'une intervention militaire, se résument aux choix réducteur de << Lindbergh ou la guerre>>. 

	De plus, quand le peuple décide de s'impliquer, il ne fait qu'étaler sa bêtise. En effet, l'absence d'esprit critique décrit ci-dessus se traduit bien souvent par le règne de la démagogie. Dans \cav, Démos, l'image du peuple, est décris comme <<sénile>>, <<à l'air débile>>. Les cavaliers lui reprochent ainsi: <<On te mène facilement et tu aimes les flatteries qui te dupent>>. Dans \adf, l'assemblé est bien vite charmée par le programme et les arguments de Proxagora, qui réussit à persuader les citoyens de confier la citée aux femmes en insistant sur les sentiments des athéniens, nostalgiques d'un temps plus ancien et moins dégénéré. Dans ces deux pièces, Aristophane tourne d'ailleurs les institutions au ridicule : On y débat le prix de l'anchois, on <<suce au Prytanée>>,  et ceux qui y participent sont décris comme <<péteurs>>, <<ivrognes>>. En plus d'être souvent incompétent, le peuple, pour l'amour de l'égalité, tend à niveler le niveau général des citoyens. En effet, Toqueville écrit dans \dda en ces temps d'égalité, << la plus petite dissemblance paraît choquante  au sein de l'uniformité générale>>.
		
	Il apparaît en fait que lorsque l'individu s'implique ou fait usage de sa liberté, c'est avant tout pour satisfaire ses intérêts personnels et non pas ceux de la communauté. On le lit clairement dans les pièces d'Aristophane : tout tourne autour de la satisfaction des besoins primaires, des bas instincts. Les premières mesures prises dans \adf par Proxagora après s'être autoproclamée dirigeante de la cité sont : organiser un grand festin, et obliger les citoyens et citoyennes les plus favorisés physiquement de satisfaire les besoins sexuelles des plus laids avant les leurs. Dans \cav, afin de gagner les faveurs de Démos, le Charcutier et le Paphlagonien le couvrent de présents, étant le plus souvent de la nourriture à foison mais aussi des coussins pour son confort et même un jeune homme de la part du Paphlagonien pour satisfaire ses désirs sexuels. Dans \cca, Lindberg président défend l'intérêt particulier dans un discours concernant les juifs: << On ne serait leur reprocher veiller sur ce qu'il considèrent comme leur intérêts, mais nous devons aussi veiller sur les nôtres. Un exemple marquant est le Rabbin Bengelsdorf, mais surtout la tante Evelyne, qui travaille activement au bureau des affaires juives et participe au dîner du président avec les représentants de l'Allemagne nazie. Aveuglée par le succès social, elle ne se rend pas compte des actes commis par le dignitaire nazi Von Ribbentrop avec qui elle danse et qu'elle trouve charmant.\\


	Mais pour autant, la tendance inverse qui consiste à imposer pour le bien de tous des mesures toujours plus extrêmes est peut-être encore plus à craindre : celle-ci reforge au contraire le peuple dans son ignorance et ses défauts et défigure la démocratie en un despotisme bien pire que tout autre forme de gouvernement. Il est alors nécessaire d'avoir confiance dans le fonctionnement de la démocratie : malgré l'imperfection des hommes, cette forme de gouvernement fonctionne et possède de réels avantages. 
	Tout d'abords, il est vrai qu'en voyant le tableau dépeint dans la première partie, l'imposition de mesures fortes semble justifiée. Ainsi, de telles décisions sont presque toujours issues d'une volonté d'amélioration. Dans \cca, la loi de peuplement Homestead 42 imaginé par le bureau d'assimilation s'affiche comme une réponse légitime et bienveillante à la question juive : les communautés juives d'Amérique renfermées sur elles même vont pouvoir s'ouvrir au reste du pays et mieux s'intégrer. De même, dans \adf, la loi sur le sexe citée précédemment à avant tout pour objectif de permettre à tous, et surtout aux plus infortunés de connaitre le plaisir charnel. Mais c'est précisément ces mesures d'égalisation, de nivellement que Toqueville dénonce. Non seulement elles détruisent la singularité des individus ( L'identité juive est dissoute et les hommes et femmes de la cité d'Athènes sont déshumanisés ), mais surtout, elles permettent à l'état de resserrer son étaut sur les individus. La loi << pénètre plus avant que jadis dans les affaires privées ; elle règle à sa manière plus d'actions, et des actions plus petites >>. Ce phénomène est en parti à l'origine de l'état pastoral despotique décrit par Toqueville. Ce <<pouvoir immense>> règne sur les individus qui n'ont plus besoin de s'intéresser aux affaires publiques ou de faire un effort quelconque. Il s'agit de l'état final de l'inaction politique. \\


	Ainsi, ce n'est qu'ayant confiance dans le processus démocratique et non pas en forçant des mesures antidémocratiques que l'on trouvera peut-être une raison pour les hommes de s'organiser démocratiquement. Car tout n'est pas à jeter dans la démocratie, et dans les faits, celle-ci ne repose pas que sur les bonnes volontés des citoyens mais sur des institutions solides qui permettent de contrer les vices humains. Dans \cca, Herman s'y raccroche. Elles représentent pour lui le dernier bastion contre l'injustice : << Il y avait Roosevelt, il y avait le Constitution des Etats-Unis, il y avais les droits civiques >> Il s'appuie aussi sur la liberté d'expression et emmène ainsi son fils Phillip au cinéma s'informer. Le rôle crucial de l'information est aussi mis en scène par le polémiste Winchell. Même si distribuer l'information est son travail, qu'il est sponsorisé par une marque de savon, il donne finalement sa vie lors d'un de ses discours publiques ayant pour but de prévenir le peuple américain du danger du gouvernement Lindbergh. L'information est aussi l'un des contre pouvoir décrit par Toquville. Il s'agit, en plus de la presse qui permet efficacement de réveiller le sentiment démocratique chez le citoyen et lui donne les moyens de se construire une pensée critique, des personnes aristocratiques : associations de citoyens leur donnant une voix commune ayant plus de chance d'être entendue. C'est aussi le bût de son livre, et même celui des pièces écrites par Aristophane : réveiller le citoyen endormi et se pencher sur les crises que peut traverser la démocratie. \\
	

	Finalement, la démocratie telle que nous la pratiquons n'est peut être pas complètement adaptée à un peuple d'hommes, elle ne le sera sans doute jamais. Mais elle met à disposition des outils lui permettant de rester en constante évolution vers un idéal meilleur. Tant que le peuple fournira l'effort minimal de toujours remettre son gouvernement en question, alors on peut le qualifier de démocratie, même si un tel gouvernement n'est peut être pas exactement l'idée que se faisait Rousseau de la démocratie. Pour conclure, même si le caractère humain semble aller à l'encontre du bon fonctionnement de la démocratie ce qui justifierait la position de Rousseau, un contrôle renforcé n'apparait pas non plus comme la solution idéale. Il apparait que la démocratie à en elle les outils pour contrer les défauts humains, reste à espérer que le peuple en fera toujours usage, et qu'ainsi les gouvernements démocratiques tendent toujours vers un idéal meilleur. 
\end{document}
