\documentclass[a4paper, 12pt]{article}
\usepackage[top=2cm, bottom=2cm, left=2.5cm, right=2.5cm]{geometry}
\usepackage{enumitem}
\usepackage{fancyhdr}
\usepackage{lmodern}
\setenumerate[1]{label=(\Roman*)}
\setenumerate[2]{label=(\arabic*)}

\begin{document}
\pagestyle{fancy}
\fancyhead[L]{Augustin Albert}
\fancyhead[R]{MPSI2 --- 2019/2020}
\fancyhead[C]{Résumé Curnier}

	Il faudrait pour certains revenir à l'essence démocratique: Nos démocraties sont molles, le pathos est valorisé  et les élites / dominent. En effet, des représentants, et non le peuple, gouvernent. Ils ne sont pas quelconques  mais choisis par le peuple / parmi une caste  d'experts  formés avant tout dans l'art de se faire élire et de remplacer le peuple. /\\

	Nos  élections sont caractéristiques d'une  oligarchie où la technique remplace la noblesse, divisant peuple et représentants. Qu'importe la / représentation, les candidats forment une caste  dont la  survie passe avant leur obligations. Seule une démocratie littérale sans division éviterait / ces écueils.\\

	La représentation serait nécessaire  car gouverner est trop complexe pour le  peuple, à l'esprit trop réducteur! Seul / des experts formée  à appréhender cette  complexité seraient capables de décrypter la situation actuelle,  de comprendre et traduire la volonté / du peuple en  actions techniquement réalisables,  mais facultatives tant le peuple est  déraisonnable... Au fond, être écouté lui suffit !\\

	La / démocratie, un entonnoir dans lequel  parlerait  le peuple ?! Insensé ! Ce n'est pas l'écoute et la compréhension qui importe  / mais la construction  d'  un système  permettant au peuple  de se diriger. La doxa entourant cette complexité révèle un clivage / entre le peuple  dont la volonté n'  est plus  pertinente face aux dirigeants plus techniciens que politiciens.\\

217 mots
\end{document}
