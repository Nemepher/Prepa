\documentclass[a4paper, 10pt]{article}
\usepackage{enumitem}
\usepackage{fancyhdr}
\usepackage[top=2cm, bottom=2cm, left=2cm, right=2cm]{geometry}

\def \hyd { H_{3}O^{+}}
\def \hox { HO^{-} }
\def \coo {C0_{2}} 
\def \eau {H_{2}O } 
\def \hhco {H_{2}C0_{3} }
\def \hco {HC0_{3}^{-} }
\def \co {C0_{3}^{2-} }
\begin{document}

\pagestyle{fancy}
\fancyhead[L]{Augustin Albert}
\fancyhead[R]{MPSI2 --- 2019/2020}

\begin{center}
\textsc{\bfseries\huge DM 8}
\end{center}
\textbf{\large Exercice 1 :}

\begin{enumerate}[label=\arabic*)]

	\item L'eau pure est non chargée donc \( [\hyd] = [\hox] = 10^{-pKe/2}\) et \( pH = -log(\hyd) = 7 \).

	\item \(\coo + \eau = \hhco \) donc à l'équilibre, \([\hhco] = K[\coo] = K\frac{n_{i}}{V} = \frac{P_{i}}{RT} = \) d'après la loi des gazs parfaits.
	\item En considerant \( [\co] \simeq [\hyd]_{0} \simeq 0 \), la réaction donnant le pH de la solution est \(\hhco + \eau = \hyd + \hco \).
		\\En notant \(\xi\) l'avancement final, \(K_{a,1}=\frac{\xi^{2}}{[\hhco]_{0} - \xi} \). 
		\\Il s'agit d'une équation polynomiale de solution \( \xi = = \). Ainsi, \(pH = -log([hyd]) = -log(\xi) = \)
	
	\item Posons \(V_{0} = 1 L\). \(  c_{tot} = [\hhco ] + [\hco] = [\hhco](1 + 10^{pH-pK_{a,1}}) = = \)\\
		\(V_{air}=\frac{c_{tot} \times V_{0}}{P_{i}}RT = = \)
	
	\item \(Ca(OH)_{2} = Ca^{2+} +2OH^{-} \) donc \(K_{s,1} = s(2s)^{2}\) et \(s=\sqrt[3]{\frac{K_{s,1}}{4}}== \)
		
		\( [Ca_{2+}]=s= \)\\
		\( [OH_{-}]=2s= \)\\
		\( pH=-log([\hyd])=-log(\frac{Ke}{[OH^{-}]})=-log()=\)
		
			
	\item Le dioxyde de carbone se retrouve sous la forme de l'acide faible \(\hyd\) dans l'eau, ce qui va acidifier la solution.
		La concentration en ion oxonium va alors diminuer et donc \( [Ca^{2+}][HO^{-}]^{2} \leq K_{s,1}\).
		Il n'y a donc pas précipitation d'hydroxyde de calcium.
	
	\item Le précipité qui va apparaitre est donc du carbonate de calcium :\\
	\[\hhco + \eau = \hco + \hyd\] \[\hco + \eau = \co + \hyd \] \[Ca^{2+} + \co = CaCO_{3}\]
		Il y aura alors précipitation si \( [\co]  \)

	\item Supposons le pH constant.\( K_{a,2}=[\co]^{2}[Ca^{2+}] \) donc \( [\co] = \sqrt{\frac{K_{a,2}}{[Ca^{2+}]}} \).
		Or, d'apres la relation de Anderson, \( [\co]=[\hco]10^{pH}K_{a,2}=[\hhco]10^{2pH}K_{a,2}K_{a,1} \) 
		d'ou \( [\hhco]=\frac{10^{-2pH}}{K_{a,2}K_{a,1}}\sqrt{\frac{K_{a,2}}{[Ca^{2+}]}}== \)
	\item En présence de c02, précipité donc mise en evidence. 
	\item C02 produit par les cellules du corps humain(respiration cellulaire?). 
	\item \(pH = pK_{a,1} + log(\frac{10^{2ko}}{10^{yolo}}) \)
	\item \([\co] = [\hco]10^{pH}pK_{a,2}\)
	\item L'acide va réagir en priorité avec la base la plus forte présente en solution. Or, les seules bases présentes sont, triées de la plus forte a la plus faible :
	\(OH^{-}, \co, Hc et A\) \[[HO^{-}]=10^{-14 +5.7} \]

	\item A est un acide donc le pH diminue.       
	\item Notant\( \xi \) l'avancement final de la réaction suivante.
	\item \( pH = pK_{a,1} + log(\frac{[base]}{[acide]})= \)
	\item D'après l'introduction, le rein permet d'eliminer des composant acie ou basique et donc fais varier le pH. 
\end{enumerate}
\end{document}
 
