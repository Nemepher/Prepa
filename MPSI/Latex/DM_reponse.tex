\documentclass[a4paper, 10pt]{article}

\usepackage{../mpsi}%il faut le fichier mpsi.sty dans le repertoire du présent fichier *.tex

\def\numeroDM{9}
\def\datederemise{jeudi 19 mars}
\def\Titre{Intégration}
\newcommand{\sujet}[1]{}
\newcommand{\reponse}[1]{\textit{#1}}

%mise en page
\usepackage[top=3cm, bottom=2cm, left=1.6cm, right=1.6cm]{geometry}
\fancyhead{}
\fancyfoot{}
\fancyhead[R]{\reponse{Un corrigé~---~}DM\numeroDM~---~\thepage/\pageref{LastPage}}
\fancyhead[L]{MPSI2~---~Lycée \textsc{Malherbe}~---~\annee}

%%Théorèmes et propositions

%type de theoreme
%\newtheoremstyle%
%{nom}% nom du style
%{espac-avant}% espace avant l'environnement
%{espace-apr�s}%espace apr�s l'environnement
%{fonte-corps}% % gras, italique, pour le corps de l'environnement
%{retrait}% retrait de la premi�re ligne
%{pont-titre}% gras, italique, etc. mais pour le titre
%{ponctuation}% c'est clair, non ?
%{espace entre le titre et le corps}
%{style-titre}% Attention, �a c'est compliqu�, mais on peut le laisser vide,

\newtheoremstyle
{exo}
{}
{}
{\large}
{}
{\bfseries\scshape}
{}
{\newline}
{\thmname{#1} \thmnumber{#2} : \thmnote{{\bf(#3)}}}

\theoremstyle{exo}
\newtheorem{exo}{Exercice}

\begin{document}
\pagestyle{fancy}

\begin{center}
\textsc{\bfseries\huge \Titre}
\end{center}
\noindent\sujet{{\small Ce DM est à me remettre le \datederemise.
Vous rédigerez ce DM en binôme.}}

\begin{exo}[Quelques intégrales impropres classiques]
  Le but du problème est de calculer les éventuelles limites suivantes, lorsque $x\to +\infty$, de :
  $$\int_0^x\e^{-t^2}\d t \et \int_0^x\frac{\sin(t)}{t}\d t.$$
  \begin{enumerate}[I.]
  \item Soient $(a,b)\in \cont{[0,1]}{\C}^2$ tel que $\re{a}$ est une fonction positive. On pose alors,
    $$F:\R_+\to\C,\,x\mapsto\int_0^1b(t)\e^{-xa(t)}\d t.$$
    On va montrer dans cette partie que $F$ est dérivable et exprimer sa dérivée sous forme intégrale.
    \begin{enumerate}[1.]
    \item Justifier que $F(x)$ est bien défini pour tout réel positif $x$.

      \reponse{Écrivez ici votre réponse sans faire de saut de ligne. Pour un retour à la ligne utiliser \texttt{\\}.$\qed$}
    \item Soit $\alpha$ un nombre complexe de partie réelle positive. En appliquant soigneusement une formule de \nom{Taylor} à la fonction $t\mapsto \e^{-\alpha t}$, montrer que
      $$\forall (x,h)\in\R^2,(x,x+h)\in,\R_+^2\Rightarrow \va{\e^{-(x+h)\alpha}-\e^{-x\alpha}+h\alpha\e^{-x\alpha}}\leq \frac{h^2}{2}\va{\alpha}^2.$$

      \reponse{}
    \item En déduire que pour tout $(x,h)\in\R^2$ tel que $[\min\left\{x,x+h\right\},\max\left\{x,x+h\right\}]\subset\R_+$,
      $$\va{\int_0^1b(t)\e^{-(x+h)a(t)}\d t-\int_0^1b(t)\e^{-xa(t)}\d t+h\int_0^1b(t)a(t)\e^{-xa(t)}\d t}\leq \frac{h^2}{2} \int_0^1\va{b(t)}\va{a(t)}^2\d t.$$

\reponse{}
  \item En déduire que $F$ est dérivable sur $\R_+$ et que pour tout réel $x$ positif,
    $$F'(x)=-\int_0^1b(t)a(t)\e^{-xa(t)}\d t.$$

\reponse{}
  \end{enumerate}
\item Dans cette partie on considère la fonction
  $$G:\R_+\to \R,\,x\mapsto \int_0^1\frac{\e^{-x(1+t^2)}}{1+t^2}\d t.$$
  \begin{enumerate}[1.]
  \item Justifier soigneusement que $G$ est dérivable sur $\R_+$ et préciser sa dérivée.

\reponse{}
  \item\label{constante} En déduire que $\disp \R_+\to \R,\,x\mapsto G\left(x^2\right)+\left(\int_0^{x}\e^{-t^2}\d t\right)^2$ est constante.

\reponse{}
  \item Montrer que pour tout réel $x$ positif,
    $$0\leq G(x^2)\leq\frac{\pi}{4} \e^{-x^2}.$$

\reponse{}
  \item En déduire très soigneusement que
    $$\int_0^x\e^{-t^2}\d t\tend{x\to+\infty}\frac{\sqrt{\pi}}{2}.$$

\reponse{}
  \end{enumerate}
\item Dans cette partie on considère la fonction
  $$H:\R_+\to \C,\,x\mapsto \int_0^1\exp\left(-x\e^{\frac{\i\pi}{2}t}\right)\d t.$$
  \begin{enumerate}[1.]
  \item\label{dérivée} Justifier soigneusement que $H$ est dérivable sur $\R_+$ et que pour tout réel strictement positif $x$,
    $$\frac{\pi}{2}H'(x)=-\frac{\sin(x)}{x}+\i\frac{\e^{-x}-\cos(x)}{x}.$$

\reponse{}
  \item Justifier que les fonctions $x\mapsto \frac{\sin(x)}{x}$ et $x\mapsto \frac{\e^{-x}-\cos(x)}{x}$ se prolongent continûment à $\R_+$. On notera $f$ et $g$ leurs prolongements continus respectifs.

\reponse{}
  \item\label{égalité} Montrer que pour tout réel positif $x$,
    $$\frac{\pi}{2}\left(H(x)-1\right)=-\int_0^xf(t)\d t +\i\int_0^xg(t)\d t.$$

\reponse{}
  \item En déduire que $H(x)\tend{x\to+\infty}0$.

\reponse{}
  \item Conclure que $\disp \int_0^xf(t)\d t\tend{x\to +\infty}\frac{\pi}{2}$.

\reponse{}
  \end{enumerate}
\end{enumerate}
\end{exo}
\end{document}
